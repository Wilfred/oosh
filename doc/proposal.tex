% Copyright Wilfred Hughes, CC-BY-SA license
\documentclass[12pt]{article}
\usepackage{a4wide} % increase page width

% todo: delete this commands
\newcommand{\al}{$<$}
\newcommand{\ar}{$>$}

\parindent 0pt
\parskip 6pt

\begin{document}

\thispagestyle{empty}

\rightline{\large Wilfred Hughes}
\medskip
\rightline{\large Churchill}
\medskip
\rightline{\large wrah2}

\vfil

\centerline{\large Part II Computer Science Project Proposal}
\vspace{0.4in}
\centerline{\Large\bf oosh: An objected oriented shell}
\vspace{0.3in}
\centerline{\large \today}

\vfil

{\bf Project Originator:} Dave Eyers

\vspace{0.1in}

{\bf Resources Required:} See attached Project Resource Form

\vspace{0.5in}

{\bf Project Supervisor:} Dave Eyers

\vspace{0.2in}

{\bf Signature:}

\vspace{0.5in}

{\bf Director of Studies:}  John Fawcett

\vspace{0.2in}

{\bf Signature:}

\vspace{0.5in}

{\bf Overseers:} Jon Crowcroft and Pietro Lio

\vspace{0.2in}

{\bf Signatures:}

\vfil
\eject

\section*{Introduction and Description of the Work}
% needs refactoring
Unix shells offer powerful ways of working with data and files using a
command line interface. Traditionally they have been written in C,
only passing text streams between programs.

I intend to extend this idea to an `object stream' where the data
passed between programs contains semantic structure and metadata. 

Due to the age of the Unix shell design (with many design decisions
visible in the original Unix shell circa 1979 [reference]) the concept
of multiple capable terminals connected to the network was not
considered. A user either ran his shell locally or dialed in to
another system and ran it there. Whilst applications (such as telnet
and ssh) have long supported file transfer they are not part of the
Unix shell abstraction.

My design will modify the basic shell design such that programs are
presented with files stored both locally and remotely, transparent to
the program performing the computation. This will make Unix commands
such as \texttt{scp} redundant as \texttt{cp} could be used in its
place.

\section*{Resources Required}
I plan to perform most of my work using my personal computer. The code
will be stored in a remote git repository so I need an Internet
connection to push my changes there regularly. If necessary I can move
to any Internet connected system that offers a Java compiler.

\section*{Starting Point}
My personal computer has run Linux for some time so I have some
expertise with Unix systems and shells. I also attended the Unix tools
lectures given by Dr Marcus Kuhn during part IB.

\section*{Substance and Structure of the Project}
The project will be written in Java due to it supporting
object-oriented programming and a safe type-system. It is also a
language I am already familiar with.

The shell itself will offer a full interpreter for shells scripts,
using a syntax similar to bash shells ([fish reference]). Posix
compliance is not necessary for this project, and other research
projects in this area have not aimed for this ([fish, previous
cambridge projects, powershell]). Java supports regular expressions so
writing a lexer will not require any specialist tools. There are also
a number of compiler-compiler tools available however a simple
recursive-descent parser should be sufficient to evaluate and run
oosh shell scripts.

I will also require a selection of programs that best demonstrate the
power of passing object streams and the advantages of network
transparency. I will implement a subset of the functionality offered
by BusyBox ([reference]), which offers a minimal shell for embedded
devices. I will also write a few small programs which manipulate
object streams to demonstrate their advantages over unstructured text streams.

\section*{Success Criteria}
The final outcome of the project should provide:

\begin{enumerate}
\item A oosh shell syntax interpreter
\item Some example oosh shell scripts
\item Location independent file access to programs run within oosh
\item Some programs which exploit this functionality
\item Some programs which exploit the fact that the object stream
  contains semantic information
\item A collection of benchmarks which show the speed difference
  between oosh and other shells
\item The dissertation must be planned and written.
\end{enumerate}

\section*{Timetable and Milestones}


\subsection*{Weeks 1 to 5}

\al\emph{Real work on the project starts here (as distinct from just
  work on the proposal).  A significant problem for Diploma candidates
  is that this critical period largely coincides with the Christmas
  vacation.  There is no guarantee that supervisors will be available
  outside Lecture Term, but Diploma students take much less of a
  Christmas break than undergraduates do, and so have some opportunity
  for uninterrupted reading and initial practical work at this stage.
  It is important to have completed some serious work on the project
  before the pressures of the Lent Term become all too apparent.}\ar

Study C and the particular implementation of it to be used.  Practise
writing various small programs, including key fragments of the compiler
and interpreter.

Milestones: Some working example C programs including code to deal
with symbol tables in the lexical analyser, and part implementation of
the time queue to be used in the interpreter/simulator.


\subsection*{Weeks 6 and 7}

Further literature study and discussion with Supervisor to ensure that
the chosen data structures are satisfactory.  Implementation of the
syntax analyser and debugging code to help test it.  This is likely to
be code that can be used to display the data structures in
human-readable form so that it is possible to check that they are as
expected.

Milestones: Ability to construct and display data structures that
represent simple ladder logic programs such as:

\begin{verbatim}
    |    A               B                C     |
    +---| |-------------| |--------------( )----+
\end{verbatim}


\subsection*{Weeks 8 to 10}

Implementation of the translation phase of the compiler. This will of
a decision to be made on what target code to generate. The obvious contnders
are: an interpretive code, Java or C.

Start to plan the Dissertation, thinking ahead especially to the
collection of examples and tests that will be used to demonstrate that
the project has been a success. 

Milestones: Ability to compile some very simple ladder logic programs,
and print out some human readable version of the target code produced.


\subsection*{Weeks 11 and 12}

Complete code for the interpreter/simulator, or the environment in
which to run the generated Java or C target code.

Prepare further test cases.  Review timetable for the remainder of the
project and adjust in the light of experience thus far.  Write the
Progress Report drawing attention to the code already written,
incorporating some examples, and recording any augmentations which at
this stage seem reasonably likely to be incorporated.

Milestones: Simple ladder logic programs should now compile and run
correctly, but probably with some serious inefficiencies in the code.
Progress Report submitted and entire project reviewed both personally
and with Overseers.


\subsection*{Weeks 13 to 19 (including Easter vacation)}

Rework the entire implementation to enhance the
richness of the language it can deal with. \al\emph{It is possible that
the initial implementation has severe restrictions on the number of
relays or coils that can be handled. Such restrictions can be freed at
this stage.}\ar\  Write initial chapters of the Dissertation.

\al\emph{The Easter break from lectures can provide a time to work on a
substantial challenge such as the computation of logarithms, where an
uninterrupted week can allow you to get to grips with a fairly
complicated algorithm.  This is a good time to put in some quiet work
(while your Supervisor is busy on other things) writing the
Preparation and Implementation chapters of the Dissertation.  By this
stage the form of the final implementation should be sufficiently
clear that most of that chapter can be written, even if the code is
incomplete.  Describing clearly what the code will do can often be a
way of sharpening your own understanding of how to implement it.}\ar

Milestones: Preparation chapter of Dissertation complete,
Implementation chapter at least half complete, code can perform a
variety of interesting tasks and should be in a state that in the
worst case it would satisfy the examiners with at most cosmetic
adjustment.


\subsection*{Weeks 20 to 26}

\al\emph{Since your project is, by now, in fairly good shape there is
a chance to use the immediate run-up to exams to attend to small
rationalisations and to implement things that are useful but fairly
straightforward.  It is generally not a good idea to drop all project
work over the revision season; if you do, the code will feel amazingly
unfamiliar when you return to it.  Equally, first priority has to go
to the exams, so do not schedule anything too demanding on the project
front here.  The fact that the Implementation chapter of the
Dissertation is in draft will mean that you should have a very clear
view of the work that remains, and so can schedule it rationally.}\ar

Work on the project will be kept ticking over during this period but
undoubtedly the Easter Term lectures and examination revision will
take priority.


\subsection*{Weeks 27 to 31}

\al\emph{Getting back to work after the examinations and May Week
  calls for discipline.  Setting a timetable can help stiffen your
  resolve!}\ar

Testing and evaluation.  Finish off otherwise ragged parts of the
code.  Write the Introduction chapter and draft the Evaluation and
Conclusions chapters of the Dissertation, complete the Implementation
chapter.

Milestones: Examples and test cases run and results collected,
Dissertation essentially complete, with large sections of it
proof-read by Supervisor and possibly friends and/or Director of
Studies.


\subsection*{Weeks 32 to 33}

Finish Dissertation, preparing diagrams for insertion.  Review whole
project, check the Dissertation, and spend a final few days on
whatever is in greatest need of attention.

\al\emph{In many cases, once a Dissertation is complete (but not
  before) it will become clear where the biggest weakness in the
  entire work is.  In some cases this will be that some feature of the
  code has not been completed or debugged, in other cases it will be
  that more sample output is needed to show the project's capabilities
  on larger test cases.  In yet other cases it will be that the
  Dissertation is not as neatly laid out or well written as would be
  ideal.  There is much to be said for reserving a small amount of
  time right at the end of the project (when your skills are most
  developed) to put in a short but intense burst of work to try to
  improve matters.  Doing this when the Dissertation is already
  complete is good: you have a clearly limited amount of time to work,
  and if your efforts fail you still have something to hand in!  If
  you succeed you may be able to replace that paragraph where you
  apologise for not getting feature X working into a brief note
  observing that you can indeed do X as well as all the other things
  you have talked about.}\ar


\subsection*{Week 34}

\al\emph{Aim to submit the dissertation at least a week before the
  deadline. Be ready to check whether you will be needed for a\/ {\rm
    viva voce} examination}.\ar

Milestone: Submission of Dissertation. 

\end{document}
