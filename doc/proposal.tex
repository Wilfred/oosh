% Copyright 2009 Wilfred Hughes, CC-BY license
\documentclass[12pt]{article}
\usepackage{a4wide} % increase page width

\parindent 0pt
\parskip 6pt

\begin{document}

\thispagestyle{empty}

\rightline{\large Wilfred Hughes}
\medskip
\rightline{\large Churchill}
\medskip
\rightline{\large wrah2}

\vfil

\centerline{\large Part II Computer Science Project Proposal}
\vspace{0.4in}
\centerline{\Large\bf oosh: An object oriented shell}
\vspace{0.3in}
\centerline{\large \today}

\vfil

{\bf Project Originator:} Wilfred Hughes

\vspace{0.1in}

{\bf Resources Required:} See attached Project Resource Form

\vspace{0.5in}

{\bf Project Supervisor:} David Eyers

\vspace{0.2in}

{\bf Signature:}

\vspace{0.5in}

{\bf Director of Studies:}  John Fawcett

\vspace{0.2in}

{\bf Signature:}

\vspace{0.5in}

{\bf Overseers:} Jon Crowcroft and Pietro Lio

\vspace{0.2in}

{\bf Signatures:}

\vfil
\eject

\section*{Introduction and Description of the Work}
% needs refactoring
Unix shells offer powerful ways of working with data and files using a
command line interface. Traditionally they have been written in C,
only passing text streams between programs.

I intend to extend this idea to an `object stream' where the data
passed between programs contains semantic structure and metadata. 

Due to the age of the Unix shell design (with many design decisions
visible in the original Unix shell circa 1979 [reference]) the concept
of multiple capable terminals connected to the network was not
considered. A user either ran his shell locally or dialled in to
another system and ran it there. Whilst applications (such as telnet
and ssh) have long supported file transfer they are not part of the
Unix shell abstraction.

My design will modify the basic shell design such that programs are
presented with files stored both locally and remotely, transparent to
the program performing the computation. This will make Unix commands
such as \texttt{scp} redundant as \texttt{cp} could be used in its
place.

\section*{Resources Required}
I plan to perform most of my work using my personal computer. The code
will be stored in a remote git repository so I need an Internet
connection to push my changes there regularly. If necessary I can move
to any Internet connected system that offers a Java compiler.

\section*{Starting Point}
My personal computer has run Linux for some time so I have some
expertise with Unix systems and shells. I also attended the Unix tools
lectures given by Dr Marcus Kuhn during part IB. I have experience in
Java due to the Java courses run in parts IA and IB, and have learnt
Emacs over the summer vacation.

\section*{Substance and Structure of the Project}
The project will be written in Java due to it supporting
object-oriented programming and a safe type-system. It is also a
language I am already familiar with.

The shell itself will offer a full interpreter for shells scripts,
using a syntax similar to bash shells ([bash syntax reference]). POSIX
compliance is not necessary for this project, and other research
projects in this area have not aimed for this ([fish, previous
Cambridge projects, powershell]). Java supports regular expressions so
writing a lexer will not require any specialist tools. There are also
a number of compiler-compiler tools available for Java programs
however a simple recursive-descent parser should be sufficient to
evaluate and run oosh shell scripts.

I will also require a selection of programs that best demonstrate the
power of passing object streams and the advantages of network
transparency. I will implement a subset of the functionality offered
by BusyBox ([reference]), which offers a minimal shell for embedded
devices. I will also write a few small programs which manipulate
object streams to demonstrate their advantages over unstructured text
streams.

\section*{Success Criteria}
The final outcome of the project should provide:

\begin{enumerate}
\item A oosh shell syntax interpreter
\item Some example oosh shell scripts
\item Location independent file access to programs run within oosh
\item Some programs which exploit this functionality
\item Some programs which exploit the fact that the object stream
  contains semantic information
\item A collection of benchmarks which show the speed difference
  between oosh and other shells
\item A clearly structured dissertation
\end{enumerate}

\section*{Timetable and Milestones}

\subsection*{Weeks 1 and 2}
Write the project proposal. Set up code repository on public facing
server. Write skeleton code and test code repository. Design and
implement object stream structure. Set up a test harness to facilitate
unit testing.

Milestones: Proposal written and submitted. Some basic code available
in git repository.

\subsection*{Weeks 3 and 4}
Research most used Unix command line tools. Select a collection to
implement, plus some others which demonstrate the power of object
streams. Implement them.

Milestones: Several commands available to be run within oosh.

\subsection*{Weeks 5 and 6}
Research Java networking facilities and develop network independence
features. Add command line tools which demonstrate its
usefulness. Start benchmarking to measure networking overhead and
tolerance to network losses.

Milestones: Commands available within oosh that use the networking
facilities. Some raw benchmark data collected.

\subsection*{Weeks 7 and 8}
Finalise oosh syntax for shell scripting. Implement
interpreter. Research most common bash shell scripts. Create
demonstration oosh shell scripts.

Milestones: Formal specification of oosh shell syntax. Example scripts
written and working.

{\bf End of Michaelmas term}.

\subsection*{Weeks 9 and 10}
Slack time. Extend test coverage. Add UI polish.

Milestones: Higher test coverage. Shell tab completion. Oosh is now
feature complete.

{\bf Beginning of Lent term}.

\subsection*{Week 11}
Write progress report. Prepare demonstration. Review timetable.

Milestones: Written and submitted progress report. Prepared and given
demonstration. Finalised timetable available.

\subsection*{Weeks 12 to 17}
Write dissertation. Write text, draw figures, prepare bibliography.

Milestones: Dissertation written and proof read by at least supervisor
and director of studies.

\subsection*{Week 18}
Solicit dissertation feedback. Exchange dissertations with other
students and proof read. Print dissertation.

Milestones: Paper copy of final dissertation draft.

{\bf End of Lent term}.

{\bf Dissertation and source code due on the 14 May 210}.

\end{document}
