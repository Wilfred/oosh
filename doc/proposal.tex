% Copyright 2009 Wilfred Hughes, CC-BY license
\documentclass[12pt]{article}
\usepackage{a4wide} % increase page width

\parindent 0pt
\parskip 6pt

\begin{document}

\thispagestyle{empty}

\rightline{\large Wilfred Hughes}
\medskip
\rightline{\large Churchill}
\medskip
\rightline{\large wrah2}

\vfil

\centerline{\large Part II Computer Science Project Proposal}
\vspace{0.4in}
\centerline{\Large\bf oosh: An object oriented shell}
\vspace{0.3in}
\centerline{\large \today}

\vfil

{\bf Project Originator:} Wilfred Hughes

\vspace{0.1in}

{\bf Resources Required:} See attached Project Resource Form

\vspace{0.5in}

{\bf Project Supervisor:} David Eyers

\vspace{0.2in}

{\bf Signature:}

\vspace{0.5in}

{\bf Director of Studies:}  John Fawcett

\vspace{0.2in}

{\bf Signature:}

\vspace{0.5in}

{\bf Overseers:} Jon Crowcroft and Pietro Lio

\vspace{0.2in}

{\bf Signatures:}

\vfil
\eject

\section*{Introduction and Description of the Work}
Unix shells offer powerful ways of working with data and files using a
command line interface. Traditionally they have been written in C,
only passing text streams between programs.

I intend to extend this idea to an `object stream' where the data
passed between programs contains semantic structure and metadata. This
will enable powerful data manipulation and processing that is not
possible with a traditional shell.

Due to the age of the Unix shell design (with many design decisions
unchanged from the original Unix shell circa 1971 \cite{firstshell})
the concept of multiple capable terminals connected to the network was
not considered. A user either ran his shell locally or dialled in to
another system and ran it there. Whilst applications (such as
\texttt{telnet} and \texttt{ssh}) have long supported file transfer
they are not part of the Unix shell abstraction.

My design will also modify the basic shell design such that programs are
presented with files stored both locally and remotely, transparent to
the program performing the computation. This will make Unix commands
such as \texttt{scp} redundant as \texttt{cp} could be used in its
place.

\section*{Resources Required}
I plan to perform most of my work using my personal computer. The code
will be stored in a remote git repository so I need an Internet
connection to push my changes there regularly. If necessary I can move
to any Internet connected system that offers a Java compiler.

\section*{Starting Point}
My personal computer has run Linux for some time so I have some
expertise with Unix systems and shells. I also attended the Unix tools
lectures given by Dr Marcus Kuhn during part IB. I have experience in
Java due to the Java courses run in parts IA and IB, and have learnt
Emacs over the summer vacation.

\section*{Substance and Structure of the Project}
The project will be written in Java due to it supporting
object-oriented programming and a safe type-system. I intend to fully
exploit the wider Java ecosystem, using the tools available to perform
tasks such as parser building and test framework development.

The shell itself will offer a full interpreter for shell scripts,
using a syntax similar to that found in bash \cite{bash}. POSIX
compliance \cite{posix} is not necessary for this project, and other
research projects in this area have not aimed for compliance to improve
readability \cite{fish} or power \cite{powershell}. I will therefore
extend the bash syntax to make shell scripts more expressive. I will
generalise the idea of utilities such as \texttt{grep} to the full
complement of the relational algebra. With the addition of metadata to
what was previously unstructured text data I can develop applications
that are content-aware; capable of modifying their behaviour based on
data format. Semantic information available also enables the
development of tools that understand the data that flows through
them. This will enable the creation of tools that visualise the data
in a natural way.

To demonstrate the additional power due to manipulating structured
data, I will also develop a selection of tools that provide
interesting data to analyse through the oosh facilities. I will
largely base this on a subset of the functionality offered by BusyBox
\cite{busybox}. Other sources of data will include the filesystem
itself and logs. This will enable the user to easily build scripts
that analyse (for example) disk space consumption over time broken
down by permissions and filetype, application logging for high CPU
applications with visualisations or nearby router performance versus
facilities offered by the router.

\section*{Success Criteria}
The final outcome of the project should provide:

\begin{enumerate}
\item A oosh shell syntax interpreter with some example scripts
\item Network transparent file access to programs run within oosh with
  some programs which exploit this
\item Some programs which exploit the fact that the object stream
  contains semantic information
\item A collection of benchmarks which show the speed difference
  between oosh and other shells
\item A clearly structured dissertation
\end{enumerate}

\section*{Timetable and Milestones}

\subsection*{Weeks 1 and 2 (8/10/2009 to 21/10/2009)}
Write the project proposal. Set up code repository on public facing
server. Write skeleton code and test code repository. Design and
implement object stream structure. Set up a test harness to facilitate
unit testing. Research Java console libraries.

Milestones: Proposal written and submitted. Some basic code available
in git repository.

\subsection*{Weeks 3 and 4 (22/10/2009 to 4/11/2009)}
Research most used Unix command line tools. Select a collection to
implement, plus some others which demonstrate the power of object
streams. Implement them.

Milestones: Several commands available to be run within oosh.

\subsection*{Weeks 5 and 6 (5/11/2009 to 18/11/2009)}
Research Java networking facilities and develop network independence
features. Add command line tools which demonstrate its
usefulness. Start benchmarking to measure networking overhead and
tolerance to network losses.

Milestones: Commands available within oosh that use the networking
facilities. Some raw benchmark data collected.

\subsection*{Weeks 7 and 8 (19/11/2009 to 2/12/2009)}
Finalise oosh syntax for shell scripting. Implement
interpreter. Research most common bash shell scripts. Create
demonstration oosh shell scripts.

Milestones: Formal specification of oosh shell syntax. Example scripts
written and working.

{\bf End of Michaelmas term}.

\subsection*{Weeks 9 to 11 (3/12/2009 to 23/12/2009)}
Slack time. Extend test coverage. Add UI polish. Implement data
visualisation facilities.

Milestones: Higher test coverage. Shell tab completion. Oosh reaches
feature completion.

{\bf Beginning of Lent term}.

\subsection*{Week 12 (14/1/2010 to 20/1/2010)}
Write progress report. Prepare demonstration. Review timetable.

Milestones: Written and submitted progress report. Prepared and given
demonstration. Finalised timetable available.

\subsection*{Weeks 13 to 15 (21/1/2010 to 10/2/2010)}
Write main body dissertation including preparation and
implementation. Write text, draw figures, prepare bibliography.

Milestones: Main body of dissertation written.

\subsection*{Weeks 16 to 18 (11/2/2010 to 3/3/2010)}
Write evaluation section of dissertation. Perform any remaining
benchmarks as necessary.

Milestones: Text of dissertation complete and proof read by at
least supervisor and director of studies.

\subsection*{Week 19 (4/3/2010 to 10/3/2010)}
Solicit dissertation feedback. Exchange dissertations with other
students and proof read. Print dissertation.

Milestones: Paper copy of final dissertation draft.

{\bf End of Lent term}.

{\bf Dissertation and source code due on the 14 May 2010}.


\begin{thebibliography}{9} % at most 9 citations

\bibitem{firstshell}
  Dennis M. Ritchie,
  \emph{The Evolution of the Unix Time-sharing System}.
  http://cm.bell-labs.com/cm/cs/who/dmr/hist.html fetched 15th October
  2009.

\bibitem{bash}
  The GNU Project,
  \emph{Bash Reference Manual}.
  http://www.gnu.org/software/bash/manual/bashref.html fetched 15th
  October 2009.
 
\bibitem{posix}
  The Open Group,
  \emph{POSIX:2008 (IEEE Std 1003.1-2008) Shell and Utilities}.
  http://www.opengroup.org/onlinepubs/9699919799/ fetched 15th October
  2009.

\bibitem{fish}
  Axel Liljencrantz,
  \emph{Fish user documentation}.
  http://fishshell.org/user\_doc/index.html fetched 15th October 2009.

\bibitem{powershell}
  Microsoft Developer Network,
  \emph{Windows Powershell Quickstart}.
  http://channel9.msdn.com/wiki/windowspowershellquickstart/ fetched
  15th October 2009.

\bibitem{busybox}
  Erik Andersen, Rob Landley and Denys Vlasenko,
  \emph{BusyBox - The Swiss Army Knife of Embedded Linux}.
  http://www.busybox.net/ fetched 15th October 2009.

\end{thebibliography}

\end{document}
