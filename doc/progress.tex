% Copyright 2009 Wilfred Hughes, CC-BY license
\documentclass[a4paper]{article}

% same formatting as proposal
\parindent 0pt
\parskip 6pt

\begin{document}

\title{Progress Report}
\author{Wilfred Hughes}
\maketitle

{\bf Project Title}: oosh: An object oriented shell

\medskip

{\bf Student}: Wilfred Hughes wrah2@cam.ac.uk

{\bf Supervisor}: David Eyers

{\bf Director of Studies}: John Fawcett

{\bf Overseers}: Jon Crowcroft and Pietro Lio

\section{Progress Relative To Schedule}
According to the original schedule, the project is two weeks behind. The
interpreter, many commands and the tabular data structure have all been
implemented and function as expected. The networking capabilities have basic
functionality implemented but the planned optimisations have not yet been
coded.

I am confident that I can recover one week in the schedule this term, and so my
projected time for completion is now one week into the Easter vacation.

\section{Difficulties Experienced}
The interpreter is a core piece of functionality and most other components of
programs interface with it in some fashion. My initial prototyping code
duplicated some functionality of the lexer and parser, resulting in internal
interfaces changing several times during development. A more efficient use of
time would have been to schedule writing the interpreter at an earlier stage.

Another complexity experienced has resulted from my choice of language, Python
3. Whilst largely enabling me to do more with less code, it is only 2 years old
and this presented a few problems. A number of libraries I wished to use were
not compatible and much third-party Python documentation is written with Python
2.X in mind. 

\section{Accomplishments So Far}
At time of writing, the shell is very capable for running commands locally. The
user may use any command accessible from a traditional shell, or he may use any
of the (currently 13) structured data commands. The two different command types
may even be interleaved at will. Oosh will print any structured data in a human
readable format and behaves as a traditional shell otherwise.

A basic client/server architecture also exists, allowing unoptimised network
distribution of commands.

Every component of Oosh understands the data structures used by structured data
commands. This enables automatic SVG graph generation based on the metadata
flowing through pipelines.

The shell also supports a number convenience functions, such as command history,
output colouring and the saving of pipeline results for further computation or
later examination.

Finally, a number of test cases have been written, ranging from simple command
tests to useful examples that would be extremely difficult to achieve with a
traditional shell.

\end{document}